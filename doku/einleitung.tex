\chapter{Einleitung}
\label{ch:intro}
\todo{Seitenränder entfernen}
\todo{Quellen}
\todo{biologischer Hintergrund genauer beschreiben}
\todo{Pluripotenz genauer Beschreiben}
Embryonale Stammzellen sind pluripotente Zellen, welche auf Grund ihrer Pluripotenz zur Neubildung von Zellgewebe genutzt werden können.
Diese Pluripotenz wird durch diverse Transkriptionsfaktoren beeinflusst. 
Außerdem sorgen diese Transkriptionsfaktoren, dafür dass diese Pluripotenz erhalten bleibt. 
Zu diesen Transkriptionsfaktoren gehören unter anderem Oct4 und Sox2, welche eine wesentlichen Effekt auf die Erhaltung der Pluripotenz haben.
Diese beiden Transkriptionsfaktoren werden von einem dritten Transkriptionsfaktor - Nanog genannt - unterstützt.


\missingfigure{Grundschema für den Ablauf aus dem Paper entnehmen.}

 
Im Zuge dieser Arbeit soll eine Anwendung entstehen, mit Hilfe welcher die Parameter, welche in Table 1 auf Seite 4 des Papers ... gelistet sind, einfach verändert werden können.
Außerdem sollen die Auswirkungen der Änderungen anhand von Graphen dargestellt werden. 
Dadurch kann überprüft werden, ob die Ergebnisse des Papers ..., vollständig nachgestellt werden können bzw. ob eine Konstellation von Parametern existiert, bei welcher die Ergebnisse widerlegt werden können. Dabei wird sich ausschließlich auf den stochastischen Versuchsansatz bezogen.

\todo{Hier Erläuterung des Papers anfügen}